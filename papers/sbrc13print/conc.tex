\section{Conclusões e trabalhos futuros}
\label{sec:conc}

A manutenção de mapas completos e atualizados da Internet é difícil
devido à grande quantidade de sondas necessárias e restrições de banda
nos monitores.  Neste trabalho propomos o \rmprt{}, uma ferramenta para
reduzir o custo de remapeamento de mudanças de roteamento na Internet.
Dadas a rota anterior a uma mudança de roteamento e um salto
(\emph{hop}) onde a mudança foi detectada, o \rmprt{} (1) realiza uma
busca binária pelo salto onde a mudança aconteceu e (2) faz o
remapeamento local da mudança.  Comparado com a abordagem atual de
remapear a nova rota por inteiro usando traceroute, o \rmprt{} reduz
significativamente o número de saltos sondados para remapear mudanças de
roteamento na Internet.  Essa redução de saltos sondados aumenta a
disponibilidade de sondas, potencializando a medição de mais caminhos ou
aumento da frequência de medições.  O remapeamento com \rmprt{} é tão
preciso quanto remapear o caminho inteiro com traceroute, e a latência
de remapeamento é satisfatória para utilização do \rmprt{} em sistemas
reais.  \rmprt{} é mais um passo na construção de mapas da topologia da
Internet mais completos e atualizados.

Como trabalho futuro, pretendemos integrar o \rmprt{} no \dtrack{} e
prover um serviço de mapeamento da Internet aberto para disponibilizar
informações sobre a topologia para pesquisadores e aplicações.  Queremos
também reduzir ainda mais o custo de remapeamento de mudanças no
\dtrack{}.  Atualmente, o \dtrack{} remapeia cada caminho separadamente.
Esta abordagem desperdiça sondas caso vários caminhos sejam afetados
pela mesma mudança.  Pretendemos desenvolver mecanismos para prever
quais caminhos são afetados por uma mesma mudança e remapear apenas um
deles.


