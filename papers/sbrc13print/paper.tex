\documentclass[12pt]{article}

\usepackage{sbc-template}
\usepackage{graphicx,url}
\usepackage[brazil, american]{babel}   
\usepackage[utf8]{inputenc}  
\usepackage{tikz}
\usepackage{subfigure}
\usepackage{amsmath}
\usepackage{amssymb}
\usepackage{cite}
% \usepackage[portugues, ruled, lined, linesnumbered]{algorithm2e}

\newcommand{\ed}[1]{\textbf{[[#1]]}}
\newcommand{\dtrack}{\textsc{dtrack}}
\newcommand{\rmprt}{{Re\-map\-Rou\-te}}
\newcommand{\figstr}{figura}
\newcommand{\secstr}{seção}
\newcommand{\secstrs}{seções}

\hyphenation{re-map-rou-te}
\hyphenation{Re-map-Rou-te}

\sloppy

\title{RemapRoute: Reduzindo o custo do remapeamento de\\
mudanças de roteamento na Internet}

\newfont{\cunhaemail}{phvr8t at 8.5pt}
\author{\'{I}talo Cunha\inst{1}\qquad Renata Teixeira\inst{2}\qquad Darryl
Veitch\inst{3}\qquad Christophe Diot\inst{4}}

\address{Departamento de Ciência da Computação, Universidade Federal
de Minas Gerais
\nextinstitute
UPMC Sorbonne Universit\'{e}s \&
conseil Nationale de la Recherche Cientifique
\nextinstitute
Department of Electrical and Electronic Engineering, University of Melbourne
\nextinstitute
Technicolor
\email{}\cunhaemail{
cunha@dcc.ufmg.br, renata.teixeira@lip6.fr,
dveitch@unimelb.edu.au, christophe.diot@technicolor.com}
}

\begin{document} 

\maketitle

\selectlanguage{american}

\begin{abstract}
%
Internet topology maps collected with traceroute may be incomplete or
out-of-date because we cannot measure frequently enough to detect all
routing changes without overloading the network.  In this paper, we show
that routing changes usually affect few routers.  We exploit this
property to design \rmprt{}, a tool to locally remap Internet routing
changes that probes only the few affected routers instead of the whole
route.  Our evaluation with trace-driven simulations and a deployment
shows that \rmprt{} significantly reduces the number of probes needed to
remap routes, with no impact on remapping accuracy or latency.  This
reduction in remapping cost allows us to build more complete and
up-to-date topology maps.
%
\end{abstract}

\selectlanguage{brazil}
     
\begin{resumo} 
%
Mapas topológicos da Internet coletados com traceroute podem estar
incompletos ou desatualizados porque não podemos realizar medições em
frequência suficiente para detectar todas as mudanças de rota.  Neste
artigo mostramos que mudanças de rota geralmente afetam poucos
roteadores.  Exploramos essa propriedade no \rmprt{}, nossa ferramenta
para remapeamento local de mudanças de roteamento que sonda apenas os
poucos roteadores afetados ao invés de toda a rota.  Nossa avaliação via
simulação e um protótipo real mostra que \rmprt{} reduz
significativamente o número de sondas de remapeamento, sem comprometer a
exatidão e a latência de remapeamento.  Esta redução do custo de
remapeamento nos propicia construir mapas topológicos mais completos e
atualizados.
%
\end{resumo}

\section{Introduction}
\label{sec:intro}

A number of distributed services and applications need to measure
Internet paths to maintain an up-to-date view of the underlying
infrastructure~\cite{duffield06binary,
dhamdhere07netdiagnoser,kompella07blackholes,
bassett12lifeguard,akamai,skitter}. All these systems use some version
of traceroute to repeatedly measure a large number of Internet paths.
Traceroute sends probes to every hop between a source and a destination,
so the measurement of a single path often requires tens of probes and
takes some seconds~\cite{veitch09balancer}. The number of probes required
to measure a path is even larger if one wants to measure all paths
between a source and a destination when routers perform load
balancing~\cite{veitch09balancer}. For example, discovering a
(multi)path with Paris traceroute, which is a traceroute version that
discovers all paths under load balancing, requires hundreds of probes
and takes tens of seconds~\cite{veitch09balancer}.\footnotemark{}
Because measuring each path takes time, a source cannot measure paths
frequently and as a result it may miss path changes. For instance,
topology mapping systems can take from several minutes to a few days to
measure all required paths~\cite{cunha11fastmapping, sherwood08discarte,
skitter}.  In between measurements, paths may be outdated or
inconsistent.

\footnotetext{
% Given the
% prevalence of load balancing in the Internet~\cite{augustin07},
It is essential to accurately measure paths under load balancing to avoid
traceroute errors and misinterpretation of path
changes~\cite{cunha11fastmapping}.}

Our previous work showed that Internet paths being mostly stable,
measuring all paths at the same frequency wastes probes in paths that
are not changing~\cite{cunha11dtrack}. We developed \dtrack{}, a system
that optimizes probing to track Internet path
changes~\cite{cunha11dtrack}.  \dtrack{} splits the task of tracking
paths into two sub-tasks.  \textit{Path change detection} sends a single
probe per path at any given time, probing more frequently paths that are
more likely to change. \textit{Path remapping} responds to a path change detection
by immediately sending probes to discover all interfaces of the new multipath.
Although \dtrack{} detects twice as many path changes
as traditional probing~\cite{cunha11dtrack}, remapping a path remains
costly.  \dtrack{} simply runs Paris traceroute to remap the entire
end-to-end path. Such complete remapping ensures the accuracy of
inferred paths, but incurs a large probing overhead.

In this paper, we show that complete remapping wastes probes because
most path changes are localized in a few (consecutive) hops of a path
(\secstr~\ref{sec:char}). We build on this observation to develop a more
efficient remapping method for \dtrack{} (\secstr~\ref{sec:remap}).
Given knowledge of the path before the change and the hop where the
change was detected, we first send probes to precisely locate the change and then
just remap the (generally few) hops that have changed. We call this
method \textit{local remapping} as opposed to the complete remapping
originally implemented in \dtrack{}. Local remapping still uses Paris
traceroute's multipath detection algorithm~\cite{veitch09balancer} for
discovering all interfaces at a given hop to guarantee accuracy, but we
no longer have to remap every single hop of the path.

Our evaluation via trace-driven simulations shows that local remapping
reduces the remapping probing cost of 88\% of path changes in our dataset by
more than half, and reduces overall cost by 73\% (\secstr~\ref{sec:sim}).
Local remapping has two  limitations: (i) it may only remap part of the
change when paths change in multiple locations; and (ii)  it does not
work when a path change reorders the interfaces of the path.  We show
that local remapping accurately remaps all changes in a path for 87\%
of path changes and that less than 1\% of path changes reorder hops
(\secstr~\ref{sec:sim}).  We develop a new version of \dtrack{} with
local remapping.  Our evaluation in a PlanetLab deployment shows that
local remapping inferences are identical to those obtained with complete
remapping for 92\% of path changes (\secstr~\ref{sec:deploy}). Yet,
local remapping reduces overall probing cost by 75\% when compared with
complete remapping.
\pagebreak

An early version of this work appeared (in Portuguese) at the Brazilian
Computer Networks and Distributed Systems Symposium~\cite{cunha13remap}.
Here we improve the path change remapping formalism and give an
evaluation with larger data sets.

% If we use the probes we save by using local remapping for detection,
% our new version of \dtrack{} can track \ed{XXX} more path changes than
% the original \dtrack{} with end-to-end remapping and \ed{XXX} more
% changes than traditional probing.

%
%Reducing remapping costs increases the number of probes available for
%topology mapping.  We can use the extra probes to monitor more paths and
%improve network coverage, or we can increase probing frequency and
%improve routing change tracking.  \rmprt{} is another step toward
%building more complete and consistent Internet maps.

%Internet failure identification systems usually require information
%about network routes~\cite{duffield06binary, dhamdhere07netdiagnoser,
%kompella07blackholes, bassett12lifeguard}.  Similarly, content
%distribution networks measure Internet routes to choose the ``best''
%server to answer a request~\cite{akamai}.  These and other systems
%measure routes frequently in an attempt to track routing changes as they
%happen.
%
%Internet route measurements are usually performed with
%traceroute~\cite{veitch09balancer, spring02rocketfuel,
%madhyastha06iplane}, which sends probes to identify a sequence of
%interfaces between a source and a destination.  The bandwidth available
%to send probes is finite.  Route measurements for topology mapping
%require a large number of probes and can take from several minutes to a
%few days~\cite{cunha11fastmapping, sherwood08discarte, skitter}.  It is
%impossible to measure routes frequently enough to detect all routing
%changes without overloading the network.  Thus, Internet topology maps
%may be outdated, or inconsistent as routing changes may occur during the
%measurement process.
%
%Our \dtrack{} system tracks Internet routing changes to keep Internet
%topology maps more up-to-date~\cite{cunha11dtrack}.  \dtrack{} separates
%the tasks of detecting and remapping routing changes.  To detect
%changes, \dtrack{} uses a lightweight probing process that combines two
%main ideas: (1) redirect probes from stable paths, where routing changes
%are unlikely, to unstable paths, where routing changes are more likely;
%and (2) spreading probes uniformly over the network and over time to
%reduce redundant probes.  \dtrack{} keeps a database with the previous
%route observed on each monitored path.  \dtrack{} sends a probe to a hop
%in the path and compares the IP address in the reply with the IP address
%in the previous route.  If the probe reply is incompatible with the
%previous route, e.g., the IP address in the probe reply is not in the
%previous route, a change is detected and the remapping process is
%triggered.
%
%To remap changes, \dtrack{} uses Paris
%traceroute~\cite{veitch09balancer} to measure the current route.  Paris
%traceroute is an improved version of traceroute that identifies routers
%that perform load balancing.  \dtrack{} uses Paris traceroute because it
%is impossible to differentiate routing changes from load balancing
%otherwise~\cite{cunha11fastmapping}.  Currently, \dtrack{} remaps the
%whole path using Paris traceroute.  This approach guarantees correct
%remapping of the current route, but wastes probes as most routing
%changes involve few hops (\secstr~\ref{sec:char}).  In particular, this
%approach ignores two pieces of information available when the remapping
%process begins: the previous route and the hop where the change was
%detected.
%
%In this paper we propose \rmprt{}, a new tool to reduce the cost of
%remapping Internet routing changes (\secstr~\ref{sec:remap}).  Given the
%previous route observed before the change and a hop where a change was
%detected, \rmprt{} strategically sends probes to locate the change and
%locally remap it rather than waste probes on an unnecessary complete
%remapping.
%
%Our evaluation via trace-driven simulations shows that \rmprt{} reduces
%by half the remapping cost of 88\% of routing changes in our dataset
%(\secstr~\ref{sec:sim}).  Remap cost reduction is even larger for routes
%traversing routers that perform load balancing.  RemapRoute's
%optimizations do not work if a previous route measurement is unavailable
%or if a routing change reorders hops (0.9\%), but is otherwise
%statistically equivalent to Paris traceroute (\secstr~\ref{sec:sim}).
%Our evaluation of \rmprt{} in a real deployment confirms our simulation
%results and demonstrates \rmprt{}'s accuracy (\secstr~\ref{sec:deploy}).
%We make the following contributions:
%
%% FIGS 1--3 FROM SEC. 3
%\begin{figure*}[t]
%\begin{minipage}{0.33\textwidth}
%\includegraphics[width=1.05\textwidth]{figs/nrouters.eps}
%\caption{Distribution of the number of hops involved in path changes.}
%\label{fig:char.nrouters}
%\end{minipage}
%\hfill
%\begin{minipage}{0.33\textwidth}
%\includegraphics[width=1.05\textwidth]{figs/fracs.eps}
%\caption{Distribution of the fraction of routers involved in path changes.}
%\label{fig:char.fracs}
%\end{minipage}
%\hfill
%\begin{minipage}{0.33\textwidth}
%\includegraphics[width=1.05\textwidth]{figs/nasns.eps}
%\caption{Distribution of the number of ASes involved in path changes.}
%\label{fig:char.nasns}
%\end{minipage}
%\end{figure*}
%
%\begin{itemize}
%%
%\item We characterize Internet routing changes and show that they
%usually involve few hops (\secstr~\ref{sec:char});
%%
%\item We propose methods to locate and remap path changes that reduce
%wasted probes (\secstr~\ref{sec:remap});
%%
%\item We show the efficacy of our tool with trace-driven simulations and
%in a real deployment (\secstrs~\ref{sec:sim} and~\ref{sec:deploy}).
%%
%\end{itemize}
%
%Reducing remapping costs increases the number of probes available for
%topology mapping.  We can use the extra probes to monitor more paths and
%improve network coverage, or we can increase probing frequency and
%improve routing change tracking.  \rmprt{} is another step toward
%building more complete and consistent Internet maps.




\section{Definições e fundamentos}
\label{sec:background}

Seguindo a nomenclatura proposta por Paxson \cite{paxson97routing},
chamamos de \emph{caminho virtual} a conectividade entre uma origem e um
destino na Internet.  Em um dado momento, um caminho virtual é
instanciado por uma \emph{rota}.  Devido a mudanças de roteamento, um
caminho pode ser visto como um processo contínuo $C(t)$ que muda de uma
rota a outra ao longo do tempo.  Uma rota é composta de \emph{saltos}
(\emph{hops}) que são instanciados por roteadores.  Saltos são
enumerados a partir da origem e nos referimos a um salto $s$ numa rota
$C(t)$ por $C(t)[s]$.  Uma rota pode ser \emph{simples} se ela tem
apenas uma sequência de roteadores da origem ao destino; ou
\emph{ramificada} se ela tem roteadores que realizam balanceamento de
carga e múltiplas sequências sobrepostas de roteadores da origem ao
destino.  Todos os saltos numa rota simples têm apenas um roteador e
pelo menos um salto numa rota ramificada tem mais de um roteador.

Dadas duas medições consecutivas de um caminho nos instantes $C(t_i)$ e
$C(t_{i+1})$, definimos uma \emph{mudança de caminho} como uma sequência
de saltos contíguos no novo caminho que altera o caminho antigo.
Computamos mudanças de caminho minimizando o número de edições (adição,
remoção e substituição de saltos) necessárias para transformar a nova
rota na rota antiga.  Definimos o \emph{salto de divergência} $s_d$ e o
\emph{salto de convergência} $s_c$ de uma mudança como os saltos
imediatamente anterior e posterior aos saltos editados pela mudança,
respectivamente.  Dizemos que o salto de divergência, o salto de
convergência e todos os saltos entre eles estão \emph{envolvidos} na
mudança de roteamento.  Exemplificando, se $C(t_i) = \{a, b, c, d, e, f,
g\}$ e $C(t_{i+1}) = \{a, b, e, x, y, g\}$, temos uma mudança com $s_d =
1$ e $s_c = 2$ (remoção de $c$ e $d$), e outra mudança com $s_d = 2$ e
$s_c = 5$ (troca de $f$ por $x$ e $y$).


\section{Path Change Characterization}
\label{sec:char}

% FIGS 1--3 FROM SEC. 3
\begin{figure*}[t]
\begin{minipage}{0.32\textwidth}
\includegraphics[width=\textwidth]{figs/nadded.eps}
\caption{Number of hops added in path changes.}
\label{fig:char.nrouters}
\end{minipage}
\hfill
%
\begin{minipage}{0.32\textwidth}
\includegraphics[width=\textwidth]{figs/fracsadded.eps}
\caption{Fraction of hops added in path changes relative to new route
length.}
\label{fig:char.fracs}
\end{minipage}
\hfill
%
\begin{minipage}{0.32\textwidth}
\includegraphics[width=\textwidth]{figs/nasns.eps}
\caption{Distribution of the number of ASes involved in path changes.}
\label{fig:char.nasns}
\end{minipage}
%
\end{figure*}

%  % OLD IMC SETUP WITH 2 FIGS ONLY.  FIGS 1--3 FROM SEC. 3
%  \begin{figure}[t]
%  %\begin{minipage}{0.33\textwidth}
%  \includegraphics[width=\columnwidth]{figs/nadded.eps}
%  \caption{Number of hops added in path changes.}
%  \label{fig:char.nrouters}
%  %\end{minipage}
%  %\hfill
%
%  %\begin{minipage}{0.33\textwidth}
%  \includegraphics[width=\columnwidth]{figs/fracsadded.eps}
%  \caption{Fraction of hops added in path changes.}
%  \label{fig:char.fracs}
%  \end{figure}
%  %\end{minipage}
%  %\hfill
%  %\begin{minipage}{0.33\textwidth}
%  %\includegraphics[width=1.05\textwidth]{figs/nasns.eps}
%  %\caption{Distribution of the number of ASes involved in path changes.}
%  %\label{fig:char.nasns}
%  %\end{minipage}


In this section we establish that most path changes involve few hops.
We deployed \dtrack{}~\cite{cunha11dtrack} (using complete remapping) to
track path changes from 72 PlanetLab nodes for one week starting March
4th, 2011.  Each monitor chose 1,000 destinations randomly from a list
of 34,820 reachable destinations.  We used a probing rate of 8 probes per
second, similar to the rate used by DIMES~\cite{shavitt09dimes}, and
observed 1,202,960  changes.  The observed paths traversed 7,315 ASes,
and 97\% of those with more than 50
customers~\cite{luckie13asrel}.

% \footnotetext{Data sets publicly available at
% www.dcc.ufmg.br/\url{~}cunha/datasets.}

%The number of hops added in a change is $h_c - h_d - 1$, assuming we know the previous route.

\figstr~\ref{fig:char.nrouters} shows the distribution of the number of hops
added by path changes, with one or more local change zones (the number of
removed hops is not shown).  This number represents the minimum number of hops
we need to measure to correctly map the new route.  We see that 78\% of
changed paths add only one or two hops, which is small compared to the median
route length of 16 hops (not shown).  The most common type of path change
(52\%) replaces one hop with another.  We note that 9\% of changes only remove
hops from the old route. This may happen when probe filtering or failures
prevent hop measurement.


% \begin{figure}[t]
% \begin{center}
% \includegraphics[width=0.8\columnwidth]{figs/nrouters.eps}
% \caption{Distribution of the number of hops involved in path changes.}
% \label{fig:char.nrouters}
% \vspace{-3mm}
% \end{center}
% \end{figure}

% \begin{figure}[t]
% \begin{center}
% \includegraphics[width=0.8\columnwidth]{figs/fracs.eps}
% \caption{Distribution of the fraction of routers involved in path
% changes.}
% \label{fig:char.fracs}
% \vspace{-3mm}
% \end{center}
% \end{figure}

\figstr~\ref{fig:char.fracs} shows the distribution of the number of
added hops as a fraction of the (new) route length.  The curve is flat
before $x = 0.033 = 1/30$ as \dtrack{} only measures up to 30 hops in a
route (the default in Paris traceroute~\cite{veitch09balancer}).  In
80\% of cases less than 18\% of hops in the new route are new.  This
result shows the potential savings from local remapping compared to
complete remapping.

We translate interface IP addresses measured by \dtrack{} to AS
numbers combining IP-to-AS mapping databases from Team
Cymru\footnotemark{} and iPlane~\cite{madhyastha06iplane}.  IP address
that do not appear in any database are given their own fake AS number,
resulting in overestimation.  \figstr~\ref{fig:char.nasns} shows the
distribution of the number of ASes involved in a given path change.  We
consider an AS to be involved if it contains any interface in a changed
hop.  We find 60\% of path changes are internal to a single AS, and only
7\% involve more than two.  The average number of hops inside each AS in
a route is 3.04.  Similarly, Paxson's seminal work on Internet routing
stability~\cite{paxson97routing} using data collected almost 20~years
ago has shown that paths are significantly more stable at the AS level
than at the IP level.  These facts reinforce the finding that changes
are local and involve few hops.

%\ed{christophe suggested looking at whether paths go back to the
%original routes after two changes.  will keep this in the queue.}

\footnotetext{Team Cymru, IP to ASN Mapping,
{http://www.team-cymru.org/}}
% Services/ip-to-asn.html}}

% \begin{figure}[t]
% \begin{center}
% \includegraphics[width=0.8\columnwidth]{figs/nasns.eps}
% \caption{Distribution of the number of ASes involved in path changes.}
% \label{fig:char.nasns}
% \end{center}
% \end{figure}


\section{Local remapping}
\label{sec:remap}

\def\Pi{p_i}
\def\Pii{p_{i-1}}

The local remapping algorithm receives as input the route observed
before the path change, $P(t_{i-1})$, and the radius $r'$ where
\dtrack{} detected a change, i.e., $P(t_i)[r'] \ne P(t_{i-1})[r']$.
Local remapping of this change involves measuring hops on the current
route $\Pi=P(t_i)$, and comparing them with hops on
$\Pii\!=\!P(t_{i-1})$.  A hop is \emph{measured} by sending multiple
probes with systematically varied IP flow-ids, similar to Paris
traceroute's MDA~\cite{veitch09balancer}, to discover the mapping
between interfaces of the measured hop and the hop before it (branch
memberships).

Local remapping operates in two phases:  (i)  locating a $\LCZ$
(\secstr~\ref{sec:remap.locate}), and (ii) remapping it
(\secstr~\ref{sec:remap.local}).  By \emph{locating} a $\LCZ$ we mean
finding a radius $r$ inside it.  If the hop at $r'$ is a changed (added)
hop, then $r'\in\LCZ(r')$ and so $\LCZ(r')$ is already located.


% Local remapping starts from radius $r'$ where the current hop differs
% from the previous one, i.e., $\Pi[r^\prime] \ne
% \Pii[r^\prime]$.  If the hop at radius $r^\prime$ in the current
% route is not contained in the previous route, i.e., $\Pi[r^\prime]
% \notin \Pii$, then radius $r^\prime$ is inside the local change
% zone, no search is necessary, and local remapping proceeds to the next
% phase to locally remap the change (\secstr~\ref{sec:remap.local}).

\subsection{Locating a LCZ}
\label{sec:remap.locate}

Assume for the moment that the only hops which have changed lie in $\LCZ(r')$, so there is
only one $\LCZ$.  By definition if we need to locate it then $r'\ge r_c$.
\figstr~\ref{fig:remap.example} gives an example of this scenario with $r'=6$.
A probe to $r'=6$ detects a path change if the returning packet comes
from interface $I_5$ instead of $\{I_6\} = \Pii[6]$.

To locate the $\LCZ$ we need to find a changed hop.  To do so we perform
a binary search over  $0\le r <r'$.  There are three possible outcomes
and associated conclusions or `rules' for the status of hop $h =\Pi[r]$
found at any $r$ during the search:

%\figstr~\ref{fig:remap.example} shows an example of a path change
%consisting of a single $\LCZ$ where hop $\{I_4\}$ was removed and hops $\{I_8\}$ and $\{I_9\}$
%added.  A probe to $r'=6$ detects a path change as the answer comes
%from $\{I_5\} = \Pi[6]$ instead of $\{I_6\} = \Pii[6]$.

\begin{description}
%
	\item[Rule 1] $h \notin \Pii$ --- conclude $r_d < r <
	r_c$, i.e.~$h$ has changed, and so is inside the local change
	zone.
%
	\item[Rule 2] $h \in \Pii$ and $\Pii\langle h\rangle
	\ne r$ --- conclude $r_c \le r$ since $h$ is unchanged but at
	different radius (hop changes must have occurred upstream).
%
	\item[Rule 3] $h\in \Pii$ and $\Pii\langle h\rangle =
	r$ --- conclude $r \le r_d$ since $h$ is unchanged, no evidence
	of any change upstream, and $\LCZ(r')$ is certainly downstream.
%
\end{description}

The location search initializes variables $r_\mathrm{up} = 0$,
$r_\mathrm{down} = r^\prime$.  At each iteration the hop at radius $r =
(r_\mathrm{up} + r_\mathrm{down})/2$ is measured and the appropriate
`Rule' applied.  Under Rule 3 we set $r_\mathrm{up} = r$; under Rule 2
we set $r_\mathrm{down} = r$.  Rule 1 implies we have located a local
change zone, so the location search is terminated and we proceed to the
actual remapping.



%The algorithm applies the following three rules after measuring hop $h =\Pi[r]$:


% Local remapping initializes $r_\mathrm{up} = 0$ and $r_\mathrm{down} =
% r^\prime$.  At each iteration in the search, local remapping measures
% the hop at radius $r = (r_\mathrm{up} + r_\mathrm{down})/2$ and
% proceeds as follows:

% If the previous route contains the hop at radius $r^\prime$ at a
% different radius $r''$, i.e., $\Pi[r^\prime] = \Pii[r'']$,
% then we have a local path change that added or removed hops to the
% path upstream of $r^\prime$.  \figstr~\ref{fig:remap.example} shows an
% example of one path change where hop $I_4$ was removed and hops $I_8$
% and $I_9$ were added.  A probe to radius six detects a path change as
% the answer comes from $\{I_5\} = \Pi[6]$ instead of $\{I_6\} =
% \Pii[6]$ in the previous route.


\begin{figure}[t] % {{{
\vspace{-7mm}
\begin{center}
\begin{tikzpicture}
\draw (-2.25,-0.5) -- (-2.25,0) -- (-2.25+0.5,0) -- (-2.25+0.5,-0.5) --
cycle;
\node at (-2,-0.25) {$s$};
%
\draw (6.25-0.5,-0.5) -- (6.25-0.5,0) -- (6.25,0) -- (6.25,-0.5) --
cycle;
\node at (6,-0.25) {$d$};
%
\draw (-1,-0.25) circle (3mm);
\draw (0,-0.25) circle (3mm);
\draw (1,-0.25) circle (3mm);
\draw[dotted, very thick] (2,-0.25) circle (3mm);
%\draw (2,-0.25) circle (3mm);
\draw (3,-0.25) circle (3mm);
\draw (4,-0.25) circle (3mm);
\draw (5,-0.25) circle (3mm);
\node at (-1,-0.25) {$I_1$};
\node at (0,-0.25) {$I_2$};
\node at (1,-0.25) {$I_3$};
\node at (2,-0.25) {$I_4$};
\node at (3,-0.25) {$I_5$};
\node at (4,-0.25) {$I_6$};
\node at (5,-0.25) {$I_7$};
%
\foreach \i in {-1,...,0}
{ \draw (\i+0.3,-0.25) -- (\i+1-0.3,-0.25); }
\foreach \i in {3,...,4}
{ \draw (\i+0.3,-0.25) -- (\i+1-0.3,-0.25); }
\draw[dotted, very thick] (1+0.3,-0.25) -- (2-0.3,-0.25);
\draw[dotted, very thick] (2+0.3,-0.25) -- (3-0.3,-0.25);
\draw (-2.25+0.5,-0.25) -- (-1-0.3,-0.25);
\draw (5+0.3,-0.25) -- (6.25-0.5,-0.25);
%
\draw[dashed, thick] (1.5,-1.25) circle (3mm);
\draw[dashed, thick] (2.5,-1.25) circle (3mm);
\node at (1.5,-1.25) {$I_8$};
\node at (2.5,-1.25) {$I_9$};
\draw[dashed, thick] (1.2,-0.45) -- (1.4,-1.05);
\draw[dashed, thick] (2.8,-0.45) -- (2.6,-1.05);
\draw[dashed, thick] (1.5+0.3,-1.25) -- (2.5-0.3,-1.25);
%
\node at (1, 0.25) {$r_d$};
\node at (3, 0.25) {$r_c$};
\node at (3, 0.75) {$r^\prime$};
%
\draw (-2,-1) -- (-1,-1) node[right] {both routes};
\draw[dashed, thick] (-2,-1.5) -- (-1,-1.5) node[right] {current route};
\draw[dotted, very thick] (-2,-2) -- (-1,-2) node[right] {previous route};
%
\end{tikzpicture}
% \vspace{3cm}
\end{center}
\vspace{-1em}
\caption{Path change removing $I_4$ and adding $I_8$ and $I_9$.}
\label{fig:remap.example}
\end{figure} % }}}

In the general case, where there are other changed hops outside
$\LCZ(r')$, the algorithm may locate another $\LCZ$ to the left of
$\LCZ(r')$ instead of $\LCZ(r')$ itself.

% and looks for hop $\Pi[h]$ on the previous route.
% Again, if hop $\Pi[h]$ is not in the previous route, i.e., $\Pi[h]
% \notin \Pii$, the search finishes and local remapping goes to the
% next phase to remap the change.  If hop $\Pi[h]$ in the current route
% is at radius $h$ in the previous route, i.e., $\Pi[h] =
% \Pii[h]$, then the path change is downstream of $h$ and local
% remapping makes $h_\mathrm{up} = h$.  If hop $\Pi[h]$ is another at
% another radius $h^{\prime\prime}$ in the previous route, i.e.,
% $\Pi[h] = \Pii[h^{\prime\prime}]$, the change is upstream of
% $h$ and local remapping makes $h_\mathrm{down} = h$.

We cannot compare the current route with the previous route if the
router at hop $\Pi[r]$ does not answer probes.  In this case we take a
conservative approach, decrementing $r$ and continuing the search at the
previous hop without updating $r_\mathrm{up}$ and $r_\mathrm{down}$.  If
a path change only removes hops, then all hops in the current route
belong to the previous route.  In this case, the search terminates when
$r_\mathrm{up} = r_\mathrm{down}$.




%%%%%%%%%%%%%%%%%%%%%%%%%%%%%%%%%%%%%%%%%%%%%%%%%%%%%
\subsection{Remapping}
\label{sec:remap.local}

Remapping starts from a radius $r$ of a changed hop inside the located
$\LCZ$, defined by $(r_d,r_c)$.  It sequentially measures hops
downstream of $r$ until it finds the (unchanged) convergence hop
$\Pi[r_c]$.  If one of the branches does not reach $d$, we define the
convergence hop as the last reachable hop, defined as one having three
following unresponsive hops, like in traceroute.  Similarly, local
remapping sequentially measures hops upstream of $r$ as needed until it
finds the (unchanged) divergence hop $\Pi[r_d]$,  terminating at the
source in the worst case.    For path changes that only remove hops, we
have $r_d = r_\mathrm{up} = r_\mathrm{down}=r_c$ at termination and no
remapping is necessary.

The algorithm operates on a principle that it will remap any changes it
becomes aware of in the course of remapping.  Thus if the radius of the
divergence hop $\Pi[r_d]$ has changed, i.e., $r_d \ne \Pii\langle
\Pi[r_d]\rangle$, then this constitutes a detection of other changes, in
particular the existence of another $\LCZ(r_d)$, defined by
$(r_d^1,r_c^1)$ with $r_c^1\le r_d$, upstream of the first.  Similarly,
hops we measured downstream of $r_c$ may have radii incompatible with
the number of hops added or removed by the zone just remapped.  In these
cases, we call the algorithm recursively starting from the radius $r''$
where the new $\LCZ(r'')$ was detected.  This process of remapping
detected changes via $\LCZ$ patches is repeated recursively until there
is no remaining evidence of change.

% \begin{algorithm}[h]
% \caption{Remap phase algorithm (\secstr~\ref{sec:remap.local})
%
% \KwIn{radius $r$ with $r_d < r < r_c$}
%
% % $r_d \leftarrow r$\textbf{,} $r_c \leftarrow r$
%
% % \textbf{while} $\Pi[r_c] \notin \Pii$\textbf{:}
% % $r_c \leftarrow r_c + 1$\textbf{,} measure($r_c$)
%
% % \textbf{while} $\Pi[r_d] \notin \Pii$\textbf{:}
% % $r_d \leftarrow r_d - 1$\textbf{,} measure($r_d$)
%
% \textbf{foreach} $r \in [r_d, r_c]$\textbf{:} measure($r$)
%
% \textbf{if} $r_d \ne \Pii\langle \Pi[r_d]\rangle$\textbf{:}
% search(0, $r_d$)
%
% \textbf{foreach} $r > r_c$ measured\textbf{:}
%
% \Indp
% $h_r \leftarrow \Pi[r]$\textbf{,} $h_c \leftarrow \Pi[r_c]$
%
% \mbox{\textbf{if} $r - r_c \ne \Pii\langle h_r\rangle -
% \Pii\langle h_c\rangle$\textbf{:} search($r_c$, $r$)}
%
% \end{algorithm}



% FIGS. 5--7 FROM SEC. 5
\begin{figure*}
\begin{minipage}{0.33\textwidth}
\includegraphics[width=1.05\textwidth]{figs/rmprtcost.eps}
\caption{Probing cost of local remapping over all possible
detection radii.}
\label{fig:sim.rmprt.start}
\end{minipage}
\hfill
\begin{minipage}{0.33\textwidth}
\includegraphics[width=1.05\textwidth]{figs/costprobe.eps}
\caption{Comparing probing cost between local and complete remapping.}
\label{fig:sim.abs.cmp}
\end{minipage}
\hfill
\begin{minipage}{0.33\textwidth}
\includegraphics[width=1.05\textwidth]{figs/costhop.eps}
\caption{Comparing number of hops measured during remapping.}
\label{fig:sim.abs.cmp.hops}
\end{minipage}
\end{figure*}


\subsection{Local remapping example}

Consider the path change shown in \figstr~\ref{fig:remap.example}, and
assume it was detected at $r'=6$ as described above.  Since hop $\{I_5\}
=\Pi[r']$ is itself unchanged (existed in $\Pii$), the detection arose
due to the radius being different than expected, 6 instead of 5, and so
$r' \notin \LCZ(r')$ and a location phase is needed.  A binary search is
therefore initiated to locate a change zone.  The search first measures
the hop at $r=r'/2=3$, and finds $\Pii[3] = \Pi[3] = \{I_3\}$,
i.e.~unchanged (and with no evidence of further changes upstream),
indicating that $r_d\ge3$.  Hence Rule 3 applies, and the search then measures
the fourth hop to find $\Pi[4] =\{I_8\}$, which is an added hop (not
in $\Pii$).  Rule 1 now applies:  a $\LCZ$ has been located at $r=4$
(with $r_d$ determined to be $r_d=3$).  The algorithm enters the
remapping phase, measures the fifth hop, and terminates having found
$(h_d,h_c)=(3,5)$ for the located $\LCZ$.  In this case the first $\LCZ$
found is $\LCZ(r')$, the only one in this route.



% \subsection{Serious local remapping example}
%
% Assume a detection at $r'=6$. The associated hop is in the old route at a different radius, so a locating phase is needed.
% Measure $r = 3$ where Rule 2 applies.
% Measure $r = 1$, find $\{I_1\}$, Rule 3 applies.
% Measure $r = 2$, find $\{I_2, I_9\}$, Rule 1 applies.
% Go to local remap and set $(r_d r_c)=(1,3)$.  Find that $\Pi[6] = \{I_5, I_8\}$ is inconsistent with
% the $\LCZ$ we have just remapped: the $\LCZ$ reduced the route length by
% one, but the radius of $\{I_5, I_8\}$ \emph{increased} by 1, implying other changes in the middle.
% Call the algorithm recursively with $r_\mathrm{up} = 3$ and $r_\mathrm{down} = 6$.
% Measure $r = 4$, find $\{I_{10}\}$,  Rule 1 applies, terminate location phase and move to remapping.
% Find $r'_d = 3$ and $r'_c = 6$.  \ed{needs a bit more polishing and checking.}
%
% \begin{figure}[t] % {{{
% \begin{center}
% \begin{tikzpicture}
% \draw (-2.25,-0.5) -- (-2.25,0) -- (-2.25+0.5,0) -- (-2.25+0.5,-0.5) --
% cycle;
% \node at (-2,-0.25) {$s$};
% %
% \draw (6.25-0.5,-0.5) -- (6.25-0.5,0) -- (6.25,0) -- (6.25,-0.5) --
% cycle;
% \node at (6,-0.25) {$d$};
% %
% \draw (-1,-0.25) circle (3mm);
% % \draw (0,-0.25) circle (3mm);
% \draw[dotted, very thick] (0,-0.25) circle (3mm);
% % \draw (1,-0.25) circle (3mm);
% \draw[dotted, very thick] (1,-0.25) circle (3mm);
% \draw (2,-0.25) circle (3mm);
% \draw (3,-0.25) ellipse (3mm and 5mm);
% \draw (4,-0.25) circle (3mm);
% \draw (5,-0.25) circle (3mm);
% \node at (-1,-0.25) {$I_1$};
% \node at (0,-0.25) {$I_2$};
% \node at (1,-0.25) {$I_3$};
% \node at (2,-0.25) {$I_4$};
% \node at (3,+0.0) {$I_5$};
% \node at (3,-0.50) {$I_8$};
% \node at (4,-0.25) {$I_6$};
% \node at (5,-0.25) {$I_7$};
% %
% % \foreach \i in {-1,...,0}
% % { \draw (\i+0.3,-0.25) -- (\i+1-0.3,-0.25); }
% \foreach \i in {3,...,4}
% { \draw (\i+0.3,-0.25) -- (\i+1-0.3,-0.25); }
% \draw[dotted, very thick] (-1+0.3,-0.25) -- (0-0.3,-0.25);
% \draw[dotted, very thick] (0+0.3,-0.25) -- (1-0.3,-0.25);
% \draw[dotted, very thick] (1+0.3,-0.25) -- (2-0.3,-0.25);
% \draw[dotted, very thick] (2+0.3,-0.25) -- (3-0.3,-0.25);
% \draw (-2.25+0.5,-0.25) -- (-1-0.3,-0.25);
% \draw (5+0.3,-0.25) -- (6.25-0.5,-0.25);
% %
% % \draw[dashed, thick] (-0.5,-1.25) circle (3mm);
% \draw[dashed, thick] (0.5,-1.25) ellipse (3mm and 5mm);
% \node at (0.5,-1.0) {$I_2$};
% \node at (0.5,-1.5) {$I_9$};
% \draw[dashed, thick] (-0.8,-0.45) -- (0.3,-1.00);
% \draw[dashed, thick] (1.8,-0.45) -- (0.7,-1.00);
% %
% \draw[dashed, thick] (2.2,-1.25) circle (3mm);
% \draw[dashed, thick] (2.6,-2.0) circle (3mm);
% \node at (2.2,-1.25) {$I_{10}$};
% \node at (2.6,-2.00) {$I_{11}$};
% \draw[dashed, thick] (2.0,-0.55) -- (2.1,-0.95);
% \draw[dashed, thick] (2.25,-1.525) -- (2.50,-1.755);
% \draw[dashed, thick] (3,-0.75) -- (2.70,-1.75);
% %
% % \node at (1, 0.25) {$r_d$};
% % \node at (3, 0.25) {$r_c$};
% % \node at (4, 0.25) {$r^\prime$};
% %
% \draw (3.25,-1.0) -- (4.25,-1.0) node[right] {both routes};
% \draw[dashed, thick] (3.25,-1.5) -- (4.25,-1.5) node[right] {current route};
% \draw[dotted, very thick] (3.25,-2.0) -- (4.25,-2.0) node[right] {previous route};
% %
% \end{tikzpicture}
% % \vspace{3cm}
% \end{center}
% \vspace{-1em}
% \caption{More complex path change.}
% \label{fig:remap.seriousexample}
% \end{figure} % }}}



\section{Trace-driven Evaluation}
\label{sec:sim}

In this section we evaluate \dtrack{} with trace-driven simulations
using the data set described in \secstr~\ref{sec:char}.  We focus
on comparing the probing cost of remapping path changes using complete
versus local remapping.

\subsection{Probing cost}

Local remapping's probing cost varies according to the radius $r^\prime$
where \dtrack{} detected the change.  \figstr~\ref{fig:sim.rmprt.start}
shows the distribution of local remapping's probing cost for the
minimum, average, and maximum costs, calculated over all radii where
\dtrack{} can detect each change.  We see that the distributions of
minimum and maximum costs are similar.  Many path changes add or remove
few hops, so there are few radii where \dtrack{} can detect the path
change and little variation in probing cost.  In the rest of this
section, we use the average cost as local remapping's probing cost.

% \begin{figure}
% \begin{center}
% \includegraphics[width=0.8\columnwidth]{figs/rmprtcost.eps}
% \caption{\rmprt{}'s remapping cost varying the hop
% $\boldsymbol{h^\prime}$ where a change is detected.}
% \label{fig:sim.rmprt.start}
% \end{center}
% \end{figure}

\figstr~\ref{fig:sim.abs.cmp} compares probing costs for both local and
complete remapping.  Despite the binary search to locate local change
zones and the need to measure convergence and divergence hops, local
remapping reduces probing costs considerably compared to complete
remapping.  \figstr~\ref{fig:sim.abs.cmp.hops} compares the number of
hops measured by the two approaches.  Comparing with
\figstr~\ref{fig:char.nrouters}, we see that local remapping frequently
measures more hops than the number of added hops, but still only a small
fraction of all hops in a route, shown by the ``complete remapping''
curve.  Local remapping measures at least three hops for 99.6\% of path
changes, even though about 9\% of path changes in our data set only
remove hops from the path (\figstr~\ref{fig:char.nrouters}).  $\LCZ$s
that only remove hops are remapped by the binary search phase of our
algorithm, which takes at least three probes except when $r' < 4$ (which
happens in 0.4\% of path changes in our data).  The remap phase in our
algorithm also measures at least three hops to remap an $\LCZ$: $r_d$,
$r_c$, and changed hops in between.

% \begin{figure}
% \begin{center}
% \includegraphics[width=0.8\columnwidth]{figs/costcmp.eps}
% \caption{Comparison of remapping cost between \rmprt{} and Paris
% traceroute.}
% \label{fig:sim.abs.cmp}
% \end{center}
% \end{figure}

\figstr~\ref{fig:sim.savings.cmp} shows the distribution of probing cost
savings when using local remapping instead of complete remapping.  We
compute probing cost savings as $(C_\mathrm{complete} -
C_\mathrm{local})/C_\mathrm{complete}$, where $C_\mathrm{complete}$ and
$C_\mathrm{local}$ are complete and local remapping's probing costs,
respectively.  The solid line, computed for all path changes in our data
set, shows that cost savings are significant.  Local remapping reduces
probing cost by more than half for 88\% of path changes, and reduces the
overall probing cost by 73\%.  The dashed lines in
\figstr~\ref{fig:sim.savings.cmp} show probing cost savings for routes
shorter than 10 hops (labelled ``short'') and routes longer than 20 hops
(labelled ``long'').  Local remapping probing cost savings is higher for
long routes, where complete remapping wastes probes in many hops that
have not changed.

% Local remapping reduces number of hops measured by more than half for
% 88\% of path changes.

\begin{figure}
\begin{center}
\includegraphics[width=0.8\columnwidth]{figs/probsavings.eps}
\caption{Probing cost savings of local remapping over complete
remapping.}
\label{fig:sim.savings.cmp}
\end{center}
\end{figure}

% FIGS. 9--11 FROM SEC. 6
\begin{figure*}
\vspace{5mm}
\begin{minipage}{0.33\textwidth}
\includegraphics[width=1.05\textwidth]{figs/ndisjoint.eps}
\caption{Number of local change zones in path changes.}
\label{fig:sim.ndisjoint}
\end{minipage}
\hfill
\begin{minipage}{0.33\textwidth}
\includegraphics[width=1.05\textwidth]{figs/probsavingsreal.eps}
\caption{Probing cost savings with local remapping in a real
deployment.}
\label{fig:deploy.savings}
\end{minipage}
\begin{minipage}{0.33\textwidth}
\includegraphics[width=1.05\textwidth]{figs/latencies.eps}
\caption{Comparing remap latency in the real deployment.}
\label{fig:deploy.latency}
\end{minipage}
\end{figure*}

\subsection{Remapping errors}
\label{sec:remap.errors}

Local remapping may only remap parts of a path change when it is
composed of multiple $\LCZ$s.  For example, a route
$p_{i-1} = [s, I_1, I_2, I_3, I_4, I_5, d]$ may change to $p_i =[s, I_1, I_6, I_3, I_9, I_5, d]$.  
In this case there are two local change zones at $(r_d,r_c)=(1,3)$ and $(r_d,r_c)=(3,5)$,
which  \dtrack{} could detect for example as $\LCZ(r'=2)$ or $\LCZ(r'=4)$. 
Local remapping could remap either depending on where the change was detected.

% \begin{minipage}{0.33\textwidth}
% \includegraphics[width=1.05\textwidth]{figs/ptrlatency.eps}
% \caption{Paris traceroute remap latency in the real deployment.}
% \label{fig:deploy.latency.ptr}
% \end{minipage}


To evaluate the prevalence of this problem,
\figstr~\ref{fig:sim.ndisjoint} shows the distribution of the number of
local change zones for consecutive path measurements in our data
set.\footnotemark{}  We observe a single $\LCZ$ in 79\% of cases; here
local remapping, just like complete remapping, will return the correct
new route regardless of detection radius $r'$ (up to statistical errors
in MDA's load balancer discovery).

% , indicating that local remapping will correctly discover the new
% route in most cases.

\footnotetext{We only measure paths when a change is detected, hence all consecutive measurements
observe at least one local change zone.}

Three other factors contribute to minimize the impact of multiple local
change zones.  First, local remapping may remap multiple
$\LCZ$s in one run.  The probability that local remapping will remap 
multiple $\LCZ$s upstream of the detection radius in our data set
is 43\% (in cases where multiple $\LCZ$s exist, not shown).  Second, an
$\LCZ$ which is not remapped when we run local remapping will be
detected and remapped in the future (assuming the new route is stable
enough).  Any $\LCZ$ that is not remapped causes only a temporary
inconsistency.  Third, probing cost savings obtained with local
remapping can be used to increase probing frequency and reduce the
chance that multiple $\LCZ$s appear.

Another limitation of local remapping is that the binary search
mechanism may fail when the relative order of two hops changes between
the previous and current routes.  An extreme, but illustrative, example
is a change from $p_{i-1} = [s, I_1, I_2, I_3, I_4, d]$ to $p_i =
[s, I_4, I_3, I_2, I_1, d]$.  Only 0.9\% of path changes in our data set
reorder hops.  As hop reordering is rare, we take the conservative
approach of remapping the whole path when we detect it. 

Finally, a change may of course occur during remapping.  
This may cause incorrect measurements regardless of remapping approach.  
In practice, \dtrack{} will correct these errors when it reprobes the path, detects that it has
changed, and consequently remaps it.


\section{Avaliação com protótipo real no PlanetLab}
\label{sec:deploy}

Nesta seção avaliamos um protótipo do \rmprt{} no PlanetLab.
Implantamos o Paris traceroute e o \rmprt{} em 140 nós PlanetLab e
coletamos medições por 18 horas de 30 de Novembro de 2012.  Cada nó
executa o Paris traceroute para medir caminhos periodicamente.  Como no
conjunto de dados utilizado nas seções anteriores, cada nó monitora
caminhos para 1.000 destinos escolhidos aleatoriamente de uma lista com
34.820 destinos alcançáveis na Internet.  Quando duas medições
consecutivas com Paris traceroute detectam uma mudança, executamos o
\rmprt{} para remapeá-la.  Cada nó leva em média 7~horas e 42~minutos
para medir os 1.000 caminhos.  Devido à baixa frequência de sondagem,
este conjunto de dados contém apenas 87.848 mudanças.  Os caminhos
medidos atravessam 7.143 sistemas autônomos e 95\% dos sistemas
autônomos de grande porte na Internet~\cite{oliveira08as2tier}.

A \figstr~\ref{fig:deploy.savings} mostra a redução do custo de
remapeamento quando usamos o \rmprt{} em vez do Paris traceroute.
Comparando com a \figstr~\ref{fig:sim.savings.cmp}, a redução média do
custo de remapeamento no cenário real é quantitativamente similar aos
resultados obtidos via simulação (linha sólida).  Por exemplo, reduzimos
pra menos da metade o custo de remapeamento de 90\% das mudanças de
caminho no cenário real.

A redução de custo para rotas curtas e longas é mais similar à redução
de custo geral no cenário real que nas simulações.  Em outras palavras,
as linhas tracejadas na \figstr~\ref{fig:deploy.savings} estão mais
próximas da linha sólida que na \figstr~\ref{fig:sim.savings.cmp}.
Atribuímos essa mudança a três fatores: (i) o menor número de mudanças
observadas no cenário real pode limitar a variedade de mudanças
observadas; (ii) diferenças no conjunto de caminhos monitorados; e (iii)
a diferente forma de detecção de mudanças (os dados utilizados nas
simulações foram coletados com o \dtrack{}).

A \figstr~\ref{fig:deploy.latency} mostra os $25^o$, $50^o$ e o $75^o$
percentis da latência de remapeamento em função do número de saltos
sondados durante o processo de remapeamento.  Avaliamos a latência de
remapeamento por que o \rmprt{} sonda saltos sequencialmente: a decisão
do próximo salto a sondar depende do resultado do último salto sondado.
O Paris traceroute, em contrapartida, poderia paralelizar a sondagem de
saltos (apesar da implementação padrão não fazê-lo).  Como a maior parte
dos remapeamentos com \rmprt{} requer sondagem de poucos saltos
(\figstr~\ref{fig:sim.abs.cmp}), a latência de remapeamento geralmente é
menor que 5 segundos.  A \figstr~\ref{fig:deploy.latency.ptr} mostra os
$25^o$, $50^o$ e $75^o$ percentis da latência de remapeamento com Paris
traceroute no cenário real.  Nosso objetivo não é comparar a latência de
remapeamento do \rmprt{} com Paris traceroute, pois a latência é
diretamente afetada por decisões de implementação da ferramenta.  Nosso
objetivo é mostrar que a latência de remapeamento com \rmprt{} é
aceitável para uso em sistemas reais.  Notamos ainda que um sistema de
mapeamento topológico como o \dtrack{} pode executar o \rmprt{}
simultaneamente em caminhos diferentes caso mais de uma mudança seja
detectada num curto intervalo de tempo.

\begin{figure}
\begin{center}
\includegraphics[width=0.47\textwidth]{figs/deploysavings.eps}
\caption{Redução do custo de remapeamento com \rmprt{} em cenários
reais.}
\label{fig:deploy.savings}
\end{center}
\end{figure}

\begin{figure}
\begin{center}
\subfigure[\rmprt{}]{
\includegraphics[width=0.47\textwidth]{figs/latency.eps}
\label{fig:deploy.latency}}
\hspace{2mm}
\subfigure[Paris traceroute]{
\includegraphics[width=0.47\textwidth]{figs/ptrlatency.eps}
\label{fig:deploy.latency.ptr}}
\caption{Latência de remapeamento em cenários reais.}
\end{center}
\end{figure}

Por último, avaliamos se o remapeamento com o \rmprt{} é equivalente a
utilizar o Paris traceroute para medir o novo caminho por inteiro.  Para
cada mudança observada no cenário real, comparamos os saltos remapeados
pelo \rmprt{} com a rota medida pelo Paris traceroute.  Apenas 0,6\% das
medições com \rmprt{} são diferentes das medições com Paris traceroute.
A identificação de roteadores que fazem balanceamento de carga usando
Paris traceroute ou \rmprt{} é probabilística~\cite{veitch09balancer}.
Por exemplo, a configuração padrão do Paris traceroute e do \rmprt{}
identifica todos os roteadores que fazem balanceamento de carga numa
rota com 95\% de confiabilidade.  Erros de medição são inevitáveis e
causam diferenças entre remapeamentos independente da ferramenta
utilizada.  Outra causa para diferença nos remapeamentos são mudanças de
roteamento que acontecem no intervalo entre a medição com Paris
traceroute e a medição com \rmprt{}.

Sumarizando, o remapeamento de mudanças de caminho na Internet com
\rmprt{} é tão preciso quanto remapeamento com Paris traceroute, reduz
significativamente o número de sondas enviadas e tem pouco impacto na
latência de remapeamento.



\section{Related work}
\label{sec:related}

% Operators can use routing protocol logs (e.g., OSPF and IS-IS) and
% router configuration files to map the topology of their
% network~\cite{turner10cenic, markopolou08sprint}.  This approach
% results in accurate and complete topologies, however it is available
% to network operators only and is restricted to a single network.

%We can use public BGP collectors\footnotemark{} and looking glass
%servers to build a map of Internet ASes~\cite{luckie13asrel,
%khan13lookinglass}.  Unfortunately, BGP does not expose all network
%links and public BGP collectors have incomplete AS
%coverage~\cite{oliveira10as2tier}.  AS-level topologies provide limited
%insight about an AS's internal topology.  In this work, we take an
%orthogonal approach and measure the network topology at the interface
%level using active probes.
%
%\footnotetext{The Univ. of Oregon Routeviews Project,
%www.routeviews.org;\newline{}
%RIPE RIS, http://www.ripe.net/data-tools/stats/ris; and others.}

Our work falls in the area of topology mapping with traceroute-style
probing. We can split research on topology mapping according to their
goals: (i) increase Internet coverage, (ii) increase the accuracy of the
measured topology, and (iii) increase measurement frequency.

The usual approach to increase network coverage is to monitor a large
number of paths.  CAIDA's Ark platform~\cite{skitter} attempts to cover
the whole Internet using few monitors to measure paths toward all
reachable $/24$ prefixes.  Unfortunately, Ark takes between two and
three days to measure all these paths due to (unavoidable) bandwidth
limitations at monitors.  An alternative is to share the probing load
among various monitors, like in DIMES~\cite{shavitt09dimes} and
Dasu~\cite{sanchez13dasu}.  These systems can directly use \dtrack{}
with local remapping to reduce network load or increase measurement
frequency.

% In this work, we assume that the sets of monitors and destinations are
% fixed.  However, \dtrack{} does not impose any restrictions on the set
% of monitors and destinations.

Techniques to increase the accuracy of the measured topology perform
more sophisticated probing.  Paris traceroute's Multipath Detection
Algorithm (MDA), for example, sends additional probes systematically
varying values of IP header fields to identify all routers that perform
load balancing in a path~\cite{veitch09balancer}.  \dtrack{} with local
remapping uses Paris traceroute's MDA algorithm to detect load
balancing.  Other tools combine traceroute with IP alias
resolution~\cite{gj13pamplona, keys13midar}, IP Record Route
probes~\cite{sherwood08discarte}, or DNS records~\cite{ferguson13dns} to
build router-level topologies.  These and other techniques send
additional probes per interface of a route and thus increase topology
mapping cost.  By identifying the few hops that have changed, local
remapping can reduce probing cost and increase probing budget available
to perform these more costly (re)mapping techniques at larger scale.

Our work is more closely related to techniques that aim at increasing the
frequency of Internet topology mapping. To increase measurement
frequency we must reduce the probing cost of each path measurement or
the number of monitored paths.  RocketFuel, for example, reduces the
cost to measure a target AS's topology by probing only one of the set of
paths that enter and exit the AS's network through the same border
routers~\cite{spring02rocketfuel}.  Another approach is to reduce the
number of destinations and probe only a single subnetwork in an
AS~\cite{beverly10hifreq}.  Tracetree~\cite{latapy08radar} assumes that
the topology from a monitor to a set of destinations is a tree so it can
eliminate redundant probes to interfaces close to the monitor.
DoubleTree reduces redundant probes to interfaces close to the monitor
(traversed by multiple paths from that monitor) and interfaces close to
the destinations (traversed by paths from different monitors toward the
same destination)~\cite{donnet05topology}. The tree assumption of
Tracetree and DoubleTree, however, is invalid in cases of load balancing
as well as under some traffic-engineering and peering practices. In
fact, our previous work~\cite{cunha11dtrack} shows that Tracetree
detects a very large number of false path changes, which \dtrack{} correctly
identifies as load balancing. \dtrack{} with local remapping reduces
probing cost by focusing on remapping only the parts of the route that
have changed while ensuring route accuracy due to its use of Paris
traceroute's MDA.




\section{Conclusions and future work}
\label{sec:conc}

In this paper we have shown that path changes in the Internet are
typically local and involve few hops.  We proposed local remapping, a
mechanism that receives the previous route and a change detection point
to efficiently remap the new route.  Local remapping first locates a
local change zone using binary search in the route, then locally remaps
the change.  We have extended \dtrack{}, our system to track path
changes, with local remapping.

Our evaluation with trace-driven simulations and in PlanetLab shows that
local remapping significantly decreases probing cost (75\% overall
reduction in the real deployment), is almost as accurate as complete
remapping (92\% of measurements are identical), and is practical for use
in real systems.  Probing savings can be used to monitor more paths and
improve network coverage, increase probing frequency to improve change
tracking, or use more elaborate probing tools (e.g., using the IP Record
Route option).

As future work, we want to identify path features that allow us to
predict if two paths share a given local change zone.  This would allow
\dtrack{} to remap a local change zone once, then recompute other paths
sharing the same local change zone immediately, without spending probes
to detect or remap the change.


\vfill\pagebreak


% \begin{algorithm}[h]
%
% \KwIn{radii $r_\mathrm{up} \le r_d$ and $r_\mathrm{down} \ge r_c$}
%
% \textbf{if} $r_\mathrm{up} \ge r_\mathrm{down}$\textbf{: return}
%
% $r \leftarrow (r_\mathrm{up} + r_\mathrm{down})/2$\textbf{,} measure($r$)
%
% \textbf{if} $r_d < r < r_c$\textbf{:} remap($r$)
%
% \textbf{else if} $r \le r_d$\textbf{:} search($r$, $r_\textrm{down}$)
%
% \textbf{else if} $r \ge r_c$\textbf{:} search($r_\mathrm{up}$, $r$)
%
% \caption{Binary search algorithm (\S~\ref{sec:remap.locate})}
% \end{algorithm}

%%%%%%%%%%%%%%%%%%%%%%%%%%%%%%%%%%%%%%%%%%%%%%%%%%%%%%%%%%%%%%%%%%%%%%%%%%%%%%
% \begin{algorithm}[h]
%
% \caption{Local remap algorithm (\secstr~\ref{sec:remap.local})}
%
% \KwIn{radius $r$ with $r_d < r < r_c$}
%
% % $r_d \leftarrow r$\textbf{,} $r_c \leftarrow r$
%
% % \textbf{while} $P(t_i)[r_c] \notin P(t_{i-1})$\textbf{:}
% % $r_c \leftarrow r_c + 1$\textbf{,} measure($r_c$)
%
% % \textbf{while} $P(t_i)[r_d] \notin P(t_{i-1})$\textbf{:}
% % $r_d \leftarrow r_d - 1$\textbf{,} measure($r_d$)
%
% \textbf{foreach} $r \in [r_d, r_c]$\textbf{:} measure($r$)
%
% \textbf{if} $r_d \ne P(t_{i-1})\langle P(t_i)[r_d]\rangle$\textbf{:}
% search(0, $r_d$)
%
% \textbf{foreach} $r > r_c$ measured\textbf{:}
%
% \Indp
% $h_r \leftarrow P(t_i)[r]$\textbf{,} $h_c \leftarrow P(t_i)[r_c]$
%
% \mbox{\textbf{if} $r - r_c \ne P(t_{i-1})\langle h_r\rangle -
% P(t_{i-1})\langle h_c\rangle$\textbf{:} search($r_c$, $r$)}
%
% \end{algorithm}



\bibliographystyle{sbc}
\bibliography{references}

\end{document}
