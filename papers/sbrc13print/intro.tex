\section{Introdução}
\label{sec:intro}

Sistemas para identificação de falhas na Internet coletam medições
frequentes de rotas na rede \cite{duffield06binary,
dhamdhere07netdiagnoser, kompella07blackholes, bassett12lifeguard}.  De
forma similar, redes de distribuição de conteúdo medem rotas na Internet
para escolher o melhor servidor para atender uma requisição
\cite{akamai}.  Esses e outros sistemas medem rotas frequentemente na
esperança de rastrear mudanças de roteamento à medida que elas
acontecem.  

Medições de rota na Internet são frequentemente coletadas com traceroute
\cite{jacobson1989traceroute, augustin07, veitch09balancer}, que envia
sondas para identificar uma sequência de roteadores entre uma origem e
um destino.  A banda disponível para enviar sondas é finita.  Medir
rotas para mapear a topologia requer um grande número de sondas e pode
levar de vários minutos a alguns dias \cite{cunha11fastmapping,
sherwood08discarte, skitter}.  É impossível medir com frequência
suficiente para detectar todas as mudanças de roteamento sem
sobrecarregar a rede.  Consequentemente, mapas da topologia da Internet
podem estar desatualizados ou inconsistentes pois mudanças de roteamento
podem acontecer durante o processo de medição.

Nosso sistema \dtrack{} rastreia mudanças de roteamento na Internet para
manter mapas da topologia da Internet mais atualizados
\cite{cunha11dtrack}.  O \dtrack{} separa as tarefas de detectar e
remapear mudanças de roteamento.  Para detectar mudanças, o \dtrack{}
usa um processo de sondagem leve que combina duas ideias: (1)
redirecionar sondas de caminhos estáveis onde mudanças de roteamento são
improváveis para caminhos instáveis onde mudanças são mais prováveis; e
(2) espalhar sondas de forma uniforme na rede para reduzir medições
redundantes.  Para remapear mudanças, o \dtrack{} usa o Paris traceroute
\cite{augustin07, veitch09balancer} para medir a nova rota por inteiro.
O Paris traceroute é uma versão moderna do traceroute capaz de
identificar roteadores que fazem balanceamento de carga.  Usamos o Paris
traceroute por que é impossível inferir mudanças de roteamento de forma
precisa sem informação sobre roteadores que fazem balanceamento de carga
\cite{cunha11fastmapping}.

O \dtrack{} mantém um banco de dados com a última rota observada em cada
um dos caminhos monitorados.  Para detectar mudanças, o \dtrack{} envia
uma sonda num ponto $s^\prime$ do caminho e compara a resposta da sonda
com a última rota observada.  Se a resposta da sonda for incompatível
com a última rota observada, e.g., o endereço IP do roteador que enviou
a resposta não pertence à última rota observada, uma mudança é detectada
e o processo de remapeamento é disparado.  Atualmente, o \dtrack{}
remapeia o caminho por inteiro usando o Paris traceroute.  Esta
abordagem garante medição correta da nova rota, mas desperdiça várias
sondas pois mudanças de caminho envolvem poucos roteadores
(\secstr~\ref{sec:char}).  Em particular, esta abordagem ignora duas
informações disponíveis quando o processo de remapeamento é disparado: a
última rota observada e o ponto $s^\prime$ onde a mudança foi detectada.

Neste artigo propomos \rmprt{}, uma ferramenta para reduzir o custo do
remapeamento de mudanças de roteamento na Internet
(seção~\ref{sec:remap}).  Dadas a última rota observada e um ponto onde
uma mudança foi detectada, o \rmprt{} envia sondas em pontos
estratégicos para localizar a mudança e remapeá-la localmente, sem
desperdiçar sondas nos roteadores que não estão envolvidos na mudança.
Nossa avaliação via simulação dirigida por dados reais mostra que
\rmprt{} reduz pela metade o custo do remapeamento de 88\% das mudanças
de roteamento em nossos dados (seção~\ref{sec:sim}).  A redução de custo
é ainda maior em rotas longas ou com roteadores que fazem balanceamento
de carga.  Nossa avaliação de um protótipo do \rmprt{} usando o
PlanetLab confirma nossos resultados via simulação e demonstra a
eficácia da ferramenta (\secstr~\ref{sec:deploy}).  Sumarizando, neste
artigo fazemos as seguintes contribuições:

\begin{itemize}

\item Caracterizamos mudanças de roteamento na Internet e mostramos que
elas envolvem uma fração pequena dos roteadores numa rota
(\secstr~\ref{sec:char});

\item Propomos métodos para localizar e remapear mudanças de roteamento
que reduzem o desperdício de sondas (\secstr~\ref{sec:remap});

\pagebreak{}
\item Mostramos que nossa ferramenta reduz o custo de remapeamento de
mudanças de rota via simulação e em cenários reais
(\secstrs~\ref{sec:sim} e~\ref{sec:deploy}).

\end{itemize}

A economia de sondas no processo de remapeamento de mudanças de
roteamento aumenta o número de sondas disponíveis para mapeamento
topológico.  Podemos utilizar estas sondas para monitorar mais caminhos
na Internet e melhorar a cobertura dos mapas da topologia, ou aumentar a
frequência de sondagem e melhorar o rastreamento de mudanças de
roteamento.  \rmprt{} é mais um passo na construção de mapas da
topologia da Internet mais completos e consistentes.


% A Internet provê o melhor serviço possível, mas não dá garantias de
% conectividade ou desempenho de ponta-a-ponta.  Apesar da Internet
% funcionar bem a maior parte do tempo, é comum problemas comprometerem
% conectividade e desempenho na Internet~\cite{athina08link,
% turner10cenic}, frustrando usuários e causando prejuízos para
% provedores de acesso.  Problemas na Internet acontecem por diversas
% razões, incluindo falhas de equipamento~\cite{bassett08hubble,
% turner10cenic}; rompimento de cabos de fibra óptica ou desastres
% naturais~\cite{quake3, quake4, quake1}; e até erros de \emph{software}
% ou erros de configuração de roteadores~\cite{feamster05nsdi}.

% A natureza distribuída da Internet implica que nenhuma entidade
% consegue solucionar problemas sozinha.  Dentro de suas redes,
% operadores podem identificar problemas usando dados e alarmes dos
% roteadores da rede~\cite{snmprfc, mahimkar08nice, athina08link}.
% Alguns problemas, infelizmente, não aparecem em sistemas de
% alarme~\cite{kompella07blackholes}.  Fora de suas redes, operadores de
% rede frequentemente contratam serviços como Keynote\footnote{Keynote
% Systems, Inc. \url{http://www.keynote.com}} ou RIPE TTM\footnote{RIPE
% Test Traffic Measurement Service.
% \url{http://www.ripe.net/data-tools/stats/ttm}} para monitorar
% conectividade e desempenho ponta-a-ponta.

% A única maneira de identificar problemas sem acesso ao equipamento de
% rede é realizar medições distribuídas de ponta-a-ponta na Internet e
% combinar essas medições para inferir o local da
% falha~\cite{nguyen07lossrate, kompella07blackholes, bassett08hubble,
% duffield06binary, dhamdhere07netdiagnoser}.  A comunidade de pesquisa
% propôs os algoritmos de tomografia de rede como uma técnica promissora
% para identificação de problemas~\cite{dhamdhere07netdiagnoser,
% duffield06binary, nguyen07lossrate}.

% Infelizmente, algoritmos de tomografia têm limitações que
% impossibilitam sua aplicação prática~\cite{huang08ccr, nguyen09tomo}.
% Em particular, algoritmos de tomografia de rede consideram que a
% topologia da rede é conhecida com exatidão.  Isto não é realista na
% Internet em geral, onde precisamos medir rotas com medições de
% ponta-a-ponta.


