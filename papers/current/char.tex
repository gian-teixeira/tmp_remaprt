\section{Path Change Characterization}
\label{sec:char}

% FIGS 1--3 FROM SEC. 3
\begin{figure*}[t]
\begin{minipage}{0.32\textwidth}
\includegraphics[width=\textwidth]{figs/nadded.eps}
\caption{Number of hops added in path changes.}
\label{fig:char.nrouters}
\end{minipage}
\hfill
%
\begin{minipage}{0.32\textwidth}
\includegraphics[width=\textwidth]{figs/fracsadded.eps}
\caption{Fraction of hops added in path changes relative to new route
length.}
\label{fig:char.fracs}
\end{minipage}
\hfill
%
\begin{minipage}{0.32\textwidth}
\includegraphics[width=\textwidth]{figs/nasns.eps}
\caption{Distribution of the number of ASes involved in path changes.}
\label{fig:char.nasns}
\end{minipage}
%
\end{figure*}

%  % OLD IMC SETUP WITH 2 FIGS ONLY.  FIGS 1--3 FROM SEC. 3
%  \begin{figure}[t]
%  %\begin{minipage}{0.33\textwidth}
%  \includegraphics[width=\columnwidth]{figs/nadded.eps}
%  \caption{Number of hops added in path changes.}
%  \label{fig:char.nrouters}
%  %\end{minipage}
%  %\hfill
%
%  %\begin{minipage}{0.33\textwidth}
%  \includegraphics[width=\columnwidth]{figs/fracsadded.eps}
%  \caption{Fraction of hops added in path changes.}
%  \label{fig:char.fracs}
%  \end{figure}
%  %\end{minipage}
%  %\hfill
%  %\begin{minipage}{0.33\textwidth}
%  %\includegraphics[width=1.05\textwidth]{figs/nasns.eps}
%  %\caption{Distribution of the number of ASes involved in path changes.}
%  %\label{fig:char.nasns}
%  %\end{minipage}


In this section we establish that most path changes involve few hops.
We deployed \dtrack{}~\cite{cunha11dtrack} (using complete remapping) to
track path changes from 72 PlanetLab nodes for one week starting March
4th, 2011.  Each monitor chose 1,000 destinations randomly from a list
of 34,820 reachable destinations.  We used a probing rate of 8 probes per
second, similar to the rate used by DIMES~\cite{shavitt09dimes}, and
observed 1,202,960  changes.  The observed paths traversed 7,315 ASes,
and 97\% of those with more than 50
customers~\cite{luckie13asrel}.

% \footnotetext{Data sets publicly available at
% www.dcc.ufmg.br/\url{~}cunha/datasets.}

%The number of hops added in a change is $h_c - h_d - 1$, assuming we know the previous route.

\figstr~\ref{fig:char.nrouters} shows the distribution of the number of hops
added by path changes, with one or more local change zones (the number of
removed hops is not shown).  This number represents the minimum number of hops
we need to measure to correctly map the new route.  We see that 78\% of
changed paths add only one or two hops, which is small compared to the median
route length of 16 hops (not shown).  The most common type of path change
(52\%) replaces one hop with another.  We note that 9\% of changes only remove
hops from the old route. This may happen when probe filtering or failures
prevent hop measurement.


% \begin{figure}[t]
% \begin{center}
% \includegraphics[width=0.8\columnwidth]{figs/nrouters.eps}
% \caption{Distribution of the number of hops involved in path changes.}
% \label{fig:char.nrouters}
% \vspace{-3mm}
% \end{center}
% \end{figure}

% \begin{figure}[t]
% \begin{center}
% \includegraphics[width=0.8\columnwidth]{figs/fracs.eps}
% \caption{Distribution of the fraction of routers involved in path
% changes.}
% \label{fig:char.fracs}
% \vspace{-3mm}
% \end{center}
% \end{figure}

\figstr~\ref{fig:char.fracs} shows the distribution of the number of
added hops as a fraction of the (new) route length.  The curve is flat
before $x = 0.033 = 1/30$ as \dtrack{} only measures up to 30 hops in a
route (the default in Paris traceroute~\cite{veitch09balancer}).  In
80\% of cases less than 18\% of hops in the new route are new.  This
result shows the potential savings from local remapping compared to
complete remapping.

We translate interface IP addresses measured by \dtrack{} to AS
numbers combining IP-to-AS mapping databases from Team
Cymru\footnotemark{} and iPlane~\cite{madhyastha06iplane}.  IP address
that do not appear in any database are given their own fake AS number,
resulting in overestimation.  \figstr~\ref{fig:char.nasns} shows the
distribution of the number of ASes involved in a given path change.  We
consider an AS to be involved if it contains any interface in a changed
hop.  We find 60\% of path changes are internal to a single AS, and only
7\% involve more than two.  The average number of hops inside each AS in
a route is 3.04.  Similarly, Paxson's seminal work on Internet routing
stability~\cite{paxson97routing} using data collected almost 20~years
ago has shown that paths are significantly more stable at the AS level
than at the IP level.  These facts reinforce the finding that changes
are local and involve few hops.

%\ed{christophe suggested looking at whether paths go back to the
%original routes after two changes.  will keep this in the queue.}

\footnotetext{Team Cymru, IP to ASN Mapping,
{http://www.team-cymru.org/}}
% Services/ip-to-asn.html}}

% \begin{figure}[t]
% \begin{center}
% \includegraphics[width=0.8\columnwidth]{figs/nasns.eps}
% \caption{Distribution of the number of ASes involved in path changes.}
% \label{fig:char.nasns}
% \end{center}
% \end{figure}
