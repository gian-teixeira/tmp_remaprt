\section{Related work}
\label{sec:related}

% Operators can use routing protocol logs (e.g., OSPF and IS-IS) and
% router configuration files to map the topology of their
% network~\cite{turner10cenic, markopolou08sprint}.  This approach
% results in accurate and complete topologies, however it is available
% to network operators only and is restricted to a single network.

%We can use public BGP collectors\footnotemark{} and looking glass
%servers to build a map of Internet ASes~\cite{luckie13asrel,
%khan13lookinglass}.  Unfortunately, BGP does not expose all network
%links and public BGP collectors have incomplete AS
%coverage~\cite{oliveira10as2tier}.  AS-level topologies provide limited
%insight about an AS's internal topology.  In this work, we take an
%orthogonal approach and measure the network topology at the interface
%level using active probes.
%
%\footnotetext{The Univ. of Oregon Routeviews Project,
%www.routeviews.org;\newline{}
%RIPE RIS, http://www.ripe.net/data-tools/stats/ris; and others.}

Our work falls in the area of topology mapping with traceroute-style
probing. We can split research on topology mapping according to their
goals: (i) increase Internet coverage, (ii) increase the accuracy of the
measured topology, and (iii) increase measurement frequency.

The usual approach to increase network coverage is to monitor a large
number of paths.  CAIDA's Ark platform~\cite{skitter} attempts to cover
the whole Internet using few monitors to measure paths toward all
reachable $/24$ prefixes.  Unfortunately, Ark takes between two and
three days to measure all these paths due to (unavoidable) bandwidth
limitations at monitors.  An alternative is to share the probing load
among various monitors, like in DIMES~\cite{shavitt09dimes} and
Dasu~\cite{sanchez13dasu}.  These systems can directly use \dtrack{}
with local remapping to reduce network load or increase measurement
frequency.

% In this work, we assume that the sets of monitors and destinations are
% fixed.  However, \dtrack{} does not impose any restrictions on the set
% of monitors and destinations.

Techniques to increase the accuracy of the measured topology perform
more sophisticated probing.  Paris traceroute's Multipath Detection
Algorithm (MDA), for example, sends additional probes systematically
varying values of IP header fields to identify all routers that perform
load balancing in a path~\cite{veitch09balancer}.  \dtrack{} with local
remapping uses Paris traceroute's MDA algorithm to detect load
balancing.  Other tools combine traceroute with IP alias
resolution~\cite{gj13pamplona, keys13midar}, IP Record Route
probes~\cite{sherwood08discarte}, or DNS records~\cite{ferguson13dns} to
build router-level topologies.  These and other techniques send
additional probes per interface of a route and thus increase topology
mapping cost.  By identifying the few hops that have changed, local
remapping can reduce probing cost and increase probing budget available
to perform these more costly (re)mapping techniques at larger scale.

Our work is more closely related to techniques that aim at increasing the
frequency of Internet topology mapping. To increase measurement
frequency we must reduce the probing cost of each path measurement or
the number of monitored paths.  RocketFuel, for example, reduces the
cost to measure a target AS's topology by probing only one of the set of
paths that enter and exit the AS's network through the same border
routers~\cite{spring02rocketfuel}.  Another approach is to reduce the
number of destinations and probe only a single subnetwork in an
AS~\cite{beverly10hifreq}.  Tracetree~\cite{latapy08radar} assumes that
the topology from a monitor to a set of destinations is a tree so it can
eliminate redundant probes to interfaces close to the monitor.
DoubleTree reduces redundant probes to interfaces close to the monitor
(traversed by multiple paths from that monitor) and interfaces close to
the destinations (traversed by paths from different monitors toward the
same destination)~\cite{donnet05topology}. The tree assumption of
Tracetree and DoubleTree, however, is invalid in cases of load balancing
as well as under some traffic-engineering and peering practices. In
fact, our previous work~\cite{cunha11dtrack} shows that Tracetree
detects a very large number of false path changes, which \dtrack{} correctly
identifies as load balancing. \dtrack{} with local remapping reduces
probing cost by focusing on remapping only the parts of the route that
have changed while ensuring route accuracy due to its use of Paris
traceroute's MDA.


