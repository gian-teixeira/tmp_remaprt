\section{Definitions and background}
\label{sec:background}

% definition of local change zone acronym
\newcommand{\LCZ}{\ensuremath\mathrm{\textsc{lcz}}}

We follow the same notation as in our previous
work~\cite{cunha11dtrack}.  We use \emph{virtual path} to refer to the
connectivity between a fixed source $s$ and a destination $d$.  At any
given time, a virtual path is realized by the \emph{current route}.
Since routing changes occur, a virtual path is a continuous time process
$P(t)$ which jumps between different routes over time. A \emph{route}
can be \emph{simple}, consisting of a sequence of IP interfaces from $s$
toward $d$, or \emph{branched}, when one or more \emph{load balancing}
routers are present, giving rise to multiple overlapping sequences
(called ``multi-paths'' in \cite{veitch09balancer}).  Thus a route is a
directed graph with interface-labelled nodes.

% \ed{A route can be a sequence that terminates before reaching $d$.
% This can occur due to routing changes (e.g., transient loops), or the
% absence of a complete route to the destination.}

By \emph{hop-set} or \emph{hop} we mean the set of interfaces (one
interface per branch), found at some fixed distance or radius from the
source.  We denote the hop-set at radius $r$ in route $p$ as $p[r]$.
Conversely, for some valid hop-set $h$ in route $p$, we let $p\langle
h\rangle$ denote the radius of $h$.  Thus $h=p[p\langle h\rangle]$.  For
example let $p =[s,I_1,I_2,\{I_3,I_4,I_5\},I_6,d]$ be a route with three
branches.  Then the 2nd hop-set is $p[2]=\{I_2\}$, the third hop is
$p[3]=\{I_3,I_4,I_5\}$, and in this route hop $\{I_6\}$ is found at
radius $p\langle \{I_6\}\rangle = 4$.  Assuming loop-free routes, hop
sets are unique, although individual interfaces can be found in multiple
hop-sets when route branches have different lengths.

%together with a specification of the flow-ids by which they can be
%reached.  By \emph{path change} we mean simply that $P(t_1)\ne P(t_2)$
%for some $t_1\ne t_2$.

For a detection method, a \emph{path change}, that is a change in the
value of $P(t)$, can only be tested for at the times $\{t_i\}$ when
measurements are made.  For  \dtrack{}, a path change is announced when
a probe test conducted at $t_i$ finds an interface at some radius $r'$
which is not a member of $P(t_{i-1})[r']$ (more accurately, of the
expected branch).  How extensive the change is is unknown. In \dtrack,
the response has been to completely remap the route by using MDA-enabled
Paris Traceroute systematically over all hops from $s$ to $d$ to obtain
an accurate $p_i = P(t_i)$ (recall that Paris Traceroute varies flow-ids
to discover each branch of a route, and in so doing associates to each
interface the flow-ids that reach it).  In this paper we try to recover
$p_i$ more cheaply by only remapping locally about radius $r'$.

%We use a hop-based notion of �local� based on finding the smallest contiguous set of changed hops centred on the detected change radius which is bounded by hop-sets which have not changed.

More precisely, we define a \emph{local change zone}  $\LCZ(r')$, based
on a detected change at radius $r'$, as the hops with radii in the range
$r_d<r<r_c$ with $r'>r_d$ in $p_i$.  Here  $r_d$, $r_c$ are the radii
respectively of the \emph{divergence} and \emph{convergence} hops,
defined as \emph{unchanged hops} obeying $r_d<r'$ bracketing the
contiguous sequence of changed hops closest to the detected change.
Note we consider a hop to be unchanged if its interface-set is
unchanged, and if the flow-ids by which each of its interfaces can be
reached have also not changed.  The radius however of the hop can
change.

For example, let $p_{i-1} = [s,I_1, I_2, I_3, I_4,I_5, \{I_6, I_7\}, d]$
and $p_i= [s,I_1, I_2, \{I_4,I_9\}, I_5, I_{10},I_{11}, d]$.  If a path
change were detected at $r'=4$, then the associated local change zone
would consist of hop $p_i[3]=\{I_4,I_9\}$, and be flanked by hop
$\{I_2\}$ at radius $r_d = 2$ and $\{l_5\}$ at $r_c=4$, because each of
these hops are unchanged from $p_{i-1}$.  Alternatively, if a path
change were detected at $r'=6$, then $\LCZ(6)$ would be specified by
$(r_d,r_c)=(4,7)$, and consist of hops $\{I_{10}\}$ and $\{I_{11}\}$.  If
however $r_d$ and $r_c$ were in reverse order to $p_{i-1}$ (i.e.,
$p_{i-1}\langle p[r_d]\rangle > p_{i-1}\langle p[r_c]\rangle$), then
$\LCZ(r')$ is not defined.  Since hops removed from $p_{i-1}$ need not
be remapped, hops in a $\LCZ$ in $p_i$ consist of new \textit{added}
hops only.

%%Example of $h'$ not in zone:
%\ed{Consider $P(t_{i-1}) = [s, I_1, I_2, I_3, I_4, d]$ and $P(t) = [s,
%I_5,I_3, I_4, d]$.  We can detect a change at radius $r^\prime = 3$,
%but the change zone is $(0, 2)$.}

%Note that `change' in the above sense relates to hops which are different to before \textbf{and} need to be remapped, i.e.~those which appear in $p_i$ but not in $p_{i-1}$, i.e.~are added.
%Hops that are  removed ($\{I_3\}$ and $\{I_4\}$ in the first example) do not need to be remapped and are therefore ignored.

