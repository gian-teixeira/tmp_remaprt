\section{Remapping Routing Events}
\label{sec:patching}

Routing events impact multiple paths in the Internet. Current
monitoring techniques monitor paths independently. Detecting
a routing event on one Internet paths does not trigger any action on
other possibly-impacted paths.  This approach (i) leads to outdated
routing information (as we do not remap paths that have possibly
changed due to the routing events) and (ii) prevents us from
observing the extent of a routing event (as other routing events
might happen before we remap all routes impacted by the first one).
In this section we investigate whether we can use information about
a just-remapped change to quickly detect and remap changes the
underlying routing event caused on other paths.  Our goal is to
develop techniques to efficiently (using few probes) identify and
remap paths impacted by a routing event.

\newcommand{\lczd}{\ensuremath\mathrm{\textsc{lczd}}}

We define a \emph{local change zone domain}, denoted $\lczd(r')$,
for a change detected at radius $r'$ as the hops removed from the
previous path, $p_{i-1}$, around $r'$. More formally, if $r_d$ and
$r_c$ are the radii of the divergence and convergence hops,
respectively, and if $r^\prime_d = p_{i-1}\langle p[r_d]\rangle$ and
$r^\prime_c = p_{i-1}\langle p[r_c]\rangle$ are the radii of the
divergence and convergence hops on the previous route, then
$\lczd(r')$ is defined as the set of hops in $p_{i-1}$ between
$r^\prime_d$ and $r^\prime_c$, i.e., $\lczd(r') = \{p_{i-1}[x]
\mid{} r^\prime_d \le x \le r^\prime_c\}$.

We extended \dtrack{} to evaluate techniques for remapping paths
after detection of a routing event.  Upon detecting a path change at
radius $r'$ on path $p_{i-1}$ (i.e., $p[r'] \ne p_{i-1}[r']$),
\dtrack{} immediately queues path $p_{i-1}$ to be remapped
(remapping starts immediately if there are no ongoing remaps).
After remapping of path $p_{i-1}$ is complete, we compute
$\lczd(r')$. Our extended \dtrack{} then queues all (other)
\emph{overlapping paths} $q$ who intersect $\lczd(r')$, i.e., $q\,
\cap\,\lczd(r') \ne \emptyset$, for remapping (if not queued). It is important to point out that this 
an overlapping path cannot enqueue other overlapping paths once this situation 
originate a recursive loop. Also, to prevent paths that change
overtime and supress other routes to enqueue overlapping paths,
we set a minimum time interval for a path to enqueue overlapping paths
(six hours). 

We deployed the extended \dtrack{} version on 80 PlanetLab nodes useing three 
sources of destinations: TOP100 Alexa,  RIPE Atlas devices and IP addresses from 
different /24 prefixes of  hitlist with 99\% of accuracy. The total number of IP 
addresses were 12763 with a coverage of 5715 ASes in IPlane database. Each PlanetLab selected
1000 random IP address from the list. We allow PlanetLab nodes have overlap
among destinations. The data used in this analysis was colleted between 01/27/2016 
to 03/07/2016, i.e., around 40 days.

Figure~\ref{fig:overlap.delay.cdf} shows the CDF, over all detected
path changes, of the approximate time it takes to remap all
overlapping paths.  Because we remap overlapping paths within
a short period, there is a lower probability that subsequent routing
events will happen while remap is ongoing.
Figure~\ref{fig:overlap.quantity.cdf} shows the CDF, over all
detected path changes, of the fraction of overlapping paths (i.e.,
fraction of other paths that overlap with local change zone
domains).  We observe that 73.68\% of all routes have at least one 
intersection which is a fuel to investigate further if these
overlapping paths also change as well and how the overlapping
interval can help us to detect the change.

\begin{figure}
\begin{center}
\includegraphics[width=0.8\columnwidth]{figs/patching/durationdetection.pdf}
\caption{CDF of detection duration to remap overlappinh routes. }
\label{fig:overlap.delay.cdf}
\end{center}
%
\end{figure}
%
\begin{figure}
\begin{center}
\includegraphics[width=0.8\columnwidth]{figs/patching/routesoverlapping.pdf}
\caption{CDF of detection overlapping with other routes.}
\label{fig:join.acc}
\end{center}
%
\end{figure}


For each detected path change, we check which of the overlapping
paths have also changed.  Figure~\ref{fig:overlap.change.prob} shows
the fraction of overlapping paths that have changed, grouping
overlapping paths by the fraction of the local change zone domain
that intersects the overlapping paths, i.e.,
$|q\,\cap\,\lczd|\div|\lczd|$.  We note that as overlapping paths
have more in common with the local change zone domain, the higher
the probability that the overlapping path will change.  

\begin{figure}
\begin{center}
\includegraphics[width=0.8\columnwidth]{figs/patching/probchange.pdf}
\caption{Probability of change given the size of the overlap. }
\label{fig:overlap.change.prob}
\end{center}
%
\end{figure}
%
%We also find
%that paths that have more in common with the path where the change
%was detected have even higher probability of changing (not shown
%\ed{@elverton: but please plot so we can see}).

We now want to find a way to identify whether an overlapping path
has changed (or remained stable).  Let $\lczd'$ denote the local
change zone domain for an overlapping path that has changed, and let
$I = \lczd\,\cap\,\lczd'$ denote the \emph{intersection} of the
local change zone domains for the path where the routing event was
detected ($\lczd$) and an overlapping path ($\lczd'$).
We also define the Candidades Probing Set ($CPS$) as the subpath 
in the overlapping path between the first and last hop in $I$.
We remove the $r_c$ and $r_d$ of $\lczd$, if they are in the overlapping,
once they are not good hops to probe in order to find changes
The data shows that if a $r_d$ is in the intersect, only 0.4\% of times
it changed in the overlapping path. Also, considering that the $r_c$ can
indicate changes when $\lczd'$ have different size, i.e., $|lczd_{p}| \ne l|czd_{p-1}| $ 
we found that the $r_c$ changed only 8.88\% in all intersects such that $r_c \in I$.
For simplicity, we define a $\lczd'_{core}$ as a $\lczd'$ with $r_d$ and $r_c$,
when the last have a $\lczd'$ with equal size, removed.


Figure~\ref{fig:lczd.intersection} shows the distribution of the
number of hops in $CPS$ that are in a $\lczd'$ by the
size of $CPS$,
i.e., $|(CPS \cap \lczd'_{core})| \div |CPS|$.  The distribution shows that the
probing almost any hop in
$\CPS$ will detect the change (if there is one). 

% NEED TO CHANGE THE LAST GRAPH: PUTTING ONLY I IN THE GRAPH?
% HOW TO DEAL WITH THE FACT THAT THERE ARE MANY OVERLAP THAT
% DOES NOT CHANGE.
%We also looked
%at where the intersection $I$ is located compared to $\lczd$.  We
%find that XX\% of the intersections $I$ are at the start of the
%local change zone (i.e., $r_d + 1$), YY\% of the intersections $I$
%are at the end of the local change zone (i.e., $r_c - 1$), and that
%only ZZ\% of the intersections do are in neither extreme of the
%change zone.


\begin{figure}
\begin{center}
\includegraphics[width=0.8\columnwidth]{figs/patching/overlapcoverage.pdf}
\caption{Probability of a probe detect a candidates that are in a LCZD on other.}
\label{fig:lczd.intersection}
\end{center}
%
\end{figure}
