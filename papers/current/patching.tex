\section{Remapping Routing Events}
\label{sec:patching}

Routing events impact multiple paths in the Internet. Current
monitoring techniques monitor paths independently. Detecting
a routing event on one Internet paths does not trigger any action on
other possibly-impacted paths.  This approach (i) leads to outdated
routing information (as we do not remap paths that have possibly
changed due to the routing events) and (ii) prevents us from
observing the extent of a routing event (as other routing events
might happen before we remap all routes impacted by the first one).
In this section we investigate whether we can use information about
a just-remapped change to quickly detect and remap changes the
underlying routing event caused on other paths.  Our goal is to
develop techniques to efficiently (using few probes) identify and
remap paths impacted by a routing event.

\newcommand{\lczd}{\ensuremath\mathrm{\textsc{lczd}}}

We define a \emph{local change zone domain}, denoted $\lczd(r')$,
for a change detected at radius $r'$ as the hops removed from the
previous path, $p_{i-1}$, around $r'$. More formally, if $r_d$ and
$r_c$ are the radii of the divergence and convergence hops,
respectively, and if $r^\prime_d = p_{i-1}\langle p[r_d]\rangle$ and
$r^\prime_c = p_{i-1}\langle p[r_c]\rangle$ are the radii of the
divergence and convergence hops on the previous route, then
$\lczd(r')$ is defined as the set of hops in $p_{i-1}$ between
$r^\prime_d$ and $r^\prime_c$, i.e., $\lczd(r') = \{p_{i-1}[x]
\mid{} r^\prime_d < x < r^\prime_c\}$.

We extended \dtrack{} to evaluate techniques for remapping paths
after detection of a routing event.  Upon detecting a path change at
radius $r'$ on path $p_{i-1}$ (i.e., $p[r'] \ne p_{i-1}[r']$),
\dtrack{} immediately queues path $p_{i-1}$ to be remapped
(remapping starts immediately if there are no ongoing remaps).
After remapping of path $p_{i-1}$ is complete, we compute
$\lczd(r')$.  Our extended \dtrack{} then queues all (other) paths
$q$ that overlap with $\lczd(r')$, i.e., $q \cap \lczd(r') \ne
\emptyset$, for remapping.  We use this data to study when and how
paths that overlap with a local change zone domain change.

Figure~\ref{fig:overlap.delay.cdf} shows the CDF, over all detected
path changes, of the approximate time it takes to remap all paths
that overlap with a local change zone domain.  Because we remap
overlapping paths within a short period, there is a lower
probability that subsequent routing events will happen while remap
is ongoing.  Figure~\ref{fig:overlap.quantity.cdf} shows the CDF,
over all detected path changes, of the fraction of other paths that
overlap with each local change zone domain.  We observe that TODO.

Given a remapped route change, Figure~\ref{fig:}






% vim: tw=68
